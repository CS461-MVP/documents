\documentclass[letterpaper, 10pt, draftclsnofoot, compsoc, onecolumn]{IEEEtran}

\usepackage{graphicx}
%\usepackage{amssymb}
%\usepackage{amsmath}
%\usepackage{amsthm}

\usepackage{alltt}
\usepackage{float}
\usepackage{url}

\usepackage{balance}
\usepackage[TABBOTCAP, tight]{subfigure}
\usepackage{enumitem}
\usepackage{pstricks, pst-node}

\usepackage{hyperref}
\usepackage{geometry}

\usepackage{comment}

\usepackage{listings}
\usepackage{color}
\usepackage[doublespacing]{setspace}
\usepackage{supertabular}
\usepackage{url}
\usepackage{fancyhdr}

%\setlength{\parindent}{4em}
%\linespread{1.1}

\pagestyle{empty}
\renewcommand{\headrulewidth}{0pt}
\lhead{Many Voices Publishing Platform}

\lfoot{TR}
\cfoot{Page}
\rfoot{\thepage}

\def\name{Steven Powers, Josh Matteson, Evan Tschuy}

%pull in the necessary preamble matter for pygments output
%\usepackage{fancyvrb}
\usepackage{color}
\usepackage[latin1]{inputenc}


\makeatletter
\def\PY@reset{\let\PY@it=\relax \let\PY@bf=\relax%
    \let\PY@ul=\relax \let\PY@tc=\relax%
    \let\PY@bc=\relax \let\PY@ff=\relax}
\def\PY@tok#1{\csname PY@tok@#1\endcsname}
\def\PY@toks#1+{\ifx\relax#1\empty\else%
    \PY@tok{#1}\expandafter\PY@toks\fi}
\def\PY@do#1{\PY@bc{\PY@tc{\PY@ul{%
    \PY@it{\PY@bf{\PY@ff{#1}}}}}}}
\def\PY#1#2{\PY@reset\PY@toks#1+\relax+\PY@do{#2}}

\expandafter\def\csname PY@tok@gd\endcsname{\def\PY@tc##1{\textcolor[rgb]{0.63,0.00,0.00}{##1}}}
\expandafter\def\csname PY@tok@gu\endcsname{\let\PY@bf=\textbf\def\PY@tc##1{\textcolor[rgb]{0.50,0.00,0.50}{##1}}}
\expandafter\def\csname PY@tok@gt\endcsname{\def\PY@tc##1{\textcolor[rgb]{0.00,0.25,0.82}{##1}}}
\expandafter\def\csname PY@tok@gs\endcsname{\let\PY@bf=\textbf}
\expandafter\def\csname PY@tok@gr\endcsname{\def\PY@tc##1{\textcolor[rgb]{1.00,0.00,0.00}{##1}}}
\expandafter\def\csname PY@tok@cm\endcsname{\let\PY@it=\textit\def\PY@tc##1{\textcolor[rgb]{0.25,0.50,0.50}{##1}}}
\expandafter\def\csname PY@tok@vg\endcsname{\def\PY@tc##1{\textcolor[rgb]{0.10,0.09,0.49}{##1}}}
\expandafter\def\csname PY@tok@m\endcsname{\def\PY@tc##1{\textcolor[rgb]{0.40,0.40,0.40}{##1}}}
\expandafter\def\csname PY@tok@mh\endcsname{\def\PY@tc##1{\textcolor[rgb]{0.40,0.40,0.40}{##1}}}
\expandafter\def\csname PY@tok@go\endcsname{\def\PY@tc##1{\textcolor[rgb]{0.50,0.50,0.50}{##1}}}
\expandafter\def\csname PY@tok@ge\endcsname{\let\PY@it=\textit}
\expandafter\def\csname PY@tok@vc\endcsname{\def\PY@tc##1{\textcolor[rgb]{0.10,0.09,0.49}{##1}}}
\expandafter\def\csname PY@tok@il\endcsname{\def\PY@tc##1{\textcolor[rgb]{0.40,0.40,0.40}{##1}}}
\expandafter\def\csname PY@tok@cs\endcsname{\let\PY@it=\textit\def\PY@tc##1{\textcolor[rgb]{0.25,0.50,0.50}{##1}}}
\expandafter\def\csname PY@tok@cp\endcsname{\def\PY@tc##1{\textcolor[rgb]{0.74,0.48,0.00}{##1}}}
\expandafter\def\csname PY@tok@gi\endcsname{\def\PY@tc##1{\textcolor[rgb]{0.00,0.63,0.00}{##1}}}
\expandafter\def\csname PY@tok@gh\endcsname{\let\PY@bf=\textbf\def\PY@tc##1{\textcolor[rgb]{0.00,0.00,0.50}{##1}}}
\expandafter\def\csname PY@tok@ni\endcsname{\let\PY@bf=\textbf\def\PY@tc##1{\textcolor[rgb]{0.60,0.60,0.60}{##1}}}
\expandafter\def\csname PY@tok@nl\endcsname{\def\PY@tc##1{\textcolor[rgb]{0.63,0.63,0.00}{##1}}}
\expandafter\def\csname PY@tok@nn\endcsname{\let\PY@bf=\textbf\def\PY@tc##1{\textcolor[rgb]{0.00,0.00,1.00}{##1}}}
\expandafter\def\csname PY@tok@no\endcsname{\def\PY@tc##1{\textcolor[rgb]{0.53,0.00,0.00}{##1}}}
\expandafter\def\csname PY@tok@na\endcsname{\def\PY@tc##1{\textcolor[rgb]{0.49,0.56,0.16}{##1}}}
\expandafter\def\csname PY@tok@nb\endcsname{\def\PY@tc##1{\textcolor[rgb]{0.00,0.50,0.00}{##1}}}
\expandafter\def\csname PY@tok@nc\endcsname{\let\PY@bf=\textbf\def\PY@tc##1{\textcolor[rgb]{0.00,0.00,1.00}{##1}}}
\expandafter\def\csname PY@tok@nd\endcsname{\def\PY@tc##1{\textcolor[rgb]{0.67,0.13,1.00}{##1}}}
\expandafter\def\csname PY@tok@ne\endcsname{\let\PY@bf=\textbf\def\PY@tc##1{\textcolor[rgb]{0.82,0.25,0.23}{##1}}}
\expandafter\def\csname PY@tok@nf\endcsname{\def\PY@tc##1{\textcolor[rgb]{0.00,0.00,1.00}{##1}}}
\expandafter\def\csname PY@tok@si\endcsname{\let\PY@bf=\textbf\def\PY@tc##1{\textcolor[rgb]{0.73,0.40,0.53}{##1}}}
\expandafter\def\csname PY@tok@s2\endcsname{\def\PY@tc##1{\textcolor[rgb]{0.73,0.13,0.13}{##1}}}
\expandafter\def\csname PY@tok@vi\endcsname{\def\PY@tc##1{\textcolor[rgb]{0.10,0.09,0.49}{##1}}}
\expandafter\def\csname PY@tok@nt\endcsname{\let\PY@bf=\textbf\def\PY@tc##1{\textcolor[rgb]{0.00,0.50,0.00}{##1}}}
\expandafter\def\csname PY@tok@nv\endcsname{\def\PY@tc##1{\textcolor[rgb]{0.10,0.09,0.49}{##1}}}
\expandafter\def\csname PY@tok@s1\endcsname{\def\PY@tc##1{\textcolor[rgb]{0.73,0.13,0.13}{##1}}}
\expandafter\def\csname PY@tok@sh\endcsname{\def\PY@tc##1{\textcolor[rgb]{0.73,0.13,0.13}{##1}}}
\expandafter\def\csname PY@tok@sc\endcsname{\def\PY@tc##1{\textcolor[rgb]{0.73,0.13,0.13}{##1}}}
\expandafter\def\csname PY@tok@sx\endcsname{\def\PY@tc##1{\textcolor[rgb]{0.00,0.50,0.00}{##1}}}
\expandafter\def\csname PY@tok@bp\endcsname{\def\PY@tc##1{\textcolor[rgb]{0.00,0.50,0.00}{##1}}}
\expandafter\def\csname PY@tok@c1\endcsname{\let\PY@it=\textit\def\PY@tc##1{\textcolor[rgb]{0.25,0.50,0.50}{##1}}}
\expandafter\def\csname PY@tok@kc\endcsname{\let\PY@bf=\textbf\def\PY@tc##1{\textcolor[rgb]{0.00,0.50,0.00}{##1}}}
\expandafter\def\csname PY@tok@c\endcsname{\let\PY@it=\textit\def\PY@tc##1{\textcolor[rgb]{0.25,0.50,0.50}{##1}}}
\expandafter\def\csname PY@tok@mf\endcsname{\def\PY@tc##1{\textcolor[rgb]{0.40,0.40,0.40}{##1}}}
\expandafter\def\csname PY@tok@err\endcsname{\def\PY@bc##1{\setlength{\fboxsep}{0pt}\fcolorbox[rgb]{1.00,0.00,0.00}{1,1,1}{\strut ##1}}}
\expandafter\def\csname PY@tok@kd\endcsname{\let\PY@bf=\textbf\def\PY@tc##1{\textcolor[rgb]{0.00,0.50,0.00}{##1}}}
\expandafter\def\csname PY@tok@ss\endcsname{\def\PY@tc##1{\textcolor[rgb]{0.10,0.09,0.49}{##1}}}
\expandafter\def\csname PY@tok@sr\endcsname{\def\PY@tc##1{\textcolor[rgb]{0.73,0.40,0.53}{##1}}}
\expandafter\def\csname PY@tok@mo\endcsname{\def\PY@tc##1{\textcolor[rgb]{0.40,0.40,0.40}{##1}}}
\expandafter\def\csname PY@tok@kn\endcsname{\let\PY@bf=\textbf\def\PY@tc##1{\textcolor[rgb]{0.00,0.50,0.00}{##1}}}
\expandafter\def\csname PY@tok@mi\endcsname{\def\PY@tc##1{\textcolor[rgb]{0.40,0.40,0.40}{##1}}}
\expandafter\def\csname PY@tok@gp\endcsname{\let\PY@bf=\textbf\def\PY@tc##1{\textcolor[rgb]{0.00,0.00,0.50}{##1}}}
\expandafter\def\csname PY@tok@o\endcsname{\def\PY@tc##1{\textcolor[rgb]{0.40,0.40,0.40}{##1}}}
\expandafter\def\csname PY@tok@kr\endcsname{\let\PY@bf=\textbf\def\PY@tc##1{\textcolor[rgb]{0.00,0.50,0.00}{##1}}}
\expandafter\def\csname PY@tok@s\endcsname{\def\PY@tc##1{\textcolor[rgb]{0.73,0.13,0.13}{##1}}}
\expandafter\def\csname PY@tok@kp\endcsname{\def\PY@tc##1{\textcolor[rgb]{0.00,0.50,0.00}{##1}}}
\expandafter\def\csname PY@tok@w\endcsname{\def\PY@tc##1{\textcolor[rgb]{0.73,0.73,0.73}{##1}}}
\expandafter\def\csname PY@tok@kt\endcsname{\def\PY@tc##1{\textcolor[rgb]{0.69,0.00,0.25}{##1}}}
\expandafter\def\csname PY@tok@ow\endcsname{\let\PY@bf=\textbf\def\PY@tc##1{\textcolor[rgb]{0.67,0.13,1.00}{##1}}}
\expandafter\def\csname PY@tok@sb\endcsname{\def\PY@tc##1{\textcolor[rgb]{0.73,0.13,0.13}{##1}}}
\expandafter\def\csname PY@tok@k\endcsname{\let\PY@bf=\textbf\def\PY@tc##1{\textcolor[rgb]{0.00,0.50,0.00}{##1}}}
\expandafter\def\csname PY@tok@se\endcsname{\let\PY@bf=\textbf\def\PY@tc##1{\textcolor[rgb]{0.73,0.40,0.13}{##1}}}
\expandafter\def\csname PY@tok@sd\endcsname{\let\PY@it=\textit\def\PY@tc##1{\textcolor[rgb]{0.73,0.13,0.13}{##1}}}

\def\PYZbs{\char`\\}
\def\PYZus{\char`\_}
\def\PYZob{\char`\{}
\def\PYZcb{\char`\}}
\def\PYZca{\char`\^}
\def\PYZam{\char`\&}
\def\PYZlt{\char`\<}
\def\PYZgt{\char`\>}
\def\PYZsh{\char`\#}
\def\PYZpc{\char`\%}
\def\PYZdl{\char`\$}
\def\PYZti{\char`\~}
% for compatibility with earlier versions
\def\PYZat{@}
\def\PYZlb{[}
\def\PYZrb{]}
\makeatother


% The following metadata will show up in the PDF properties
\hypersetup{
  colorlinks=true, linkcolor=blue,citecolor=blue, filecolor=blue,
  urlcolor=blue, pdfauthor = {\name},
  pdfkeywords = {cs461 ``senior capstone'' technology review},
  pdftitle = {CS 461 Technology Review},
  pdfsubject = {CS 461 Technology Review},
  pdfpagemode = UseNone
}





\begin{document}


\begin{titlepage}
\centering
{\huge Many Voices Publishing Platform\par}
{\LARGE Technology Review\par}
{\vspace{5mm}}
{\large D. Kevin McGrath \& Dr. Kirsten Winters -  CS461 Fall 2016\par}
{\large Commix\par}
{\large Steven Powers, Josh Matteson, Evan Tschuy\par}
{\vspace{10mm}}

\begin{abstract}
\noindent The Many Voices Publishing Platform uses a variety of technologies to handle different aspects of the project, from the user interface to the backend database operations. These technologies enable to the Many Voices Publishing Platform to succeed in delivering a working platform for textbook collaboration.
\end{abstract}

\end{titlepage}

\tableofcontents

\newpage

\setcounter{page}{1}\pagestyle{fancy}

\vspace{1pc}
\section{Technology Review}

\begin{comment}


I. Introduction to entire tech review, including authorship of each section

II.  Technologies 1-9 (each person responsible for 3)

(1)     Technology 1

a.       Options 1, 2, and 3

b.       Goals for use in design

c.       Criteria being evaluated (e.g., cost, availability, speed, security, etc)

d.       Table comparing option 1, option 2, option 3 based on criteria

e.       Discussion

f.        Selection of best option based on criteria

(2)     Technology 2

a.       Options 1, 2, and 3

b.       Goals for use in design

c.       Criteria being evaluated (e.g., cost, availability, speed, security, etc)

d.       Table comparing option 1, option 2, option 3 based on criteria

e.       Discussion

f.        Selection of best option based on criteria

(3)     Technology 3

a.       Options 1, 2, and 3

b.       Goals

c.       Criteria

d.       Table

e.       Discussion

f.        Selection of best option based on criteria

(4)     ….etc….

III.                Conclusion

IV.                Bibliography

\end{comment}
\vspace{1pc}
\subsection{Introduction}
\vspace{1pc}

{\noindent The Many Voices Publishing Platform is being developed for the purpose of fixing the problems currently associated with the textbook market. We will accomplish this by giving the MVP Platform an easy to use interface, a search bar with a built in results pane, source control, and many other features. Authorship is divided by subsection header.\par}




\subsection{Steven}
\vspace{1pc}

{\subsubsection{User Interface Tools}
{\noindent Option 1 - React \cite{React} \par}
{\noindent React is a JavaScript rendering engine that is developed by Facebook. Originally used with Instagram, React is often paired with Redux for added functionality. React is a popular JavaScript library meant for building user interfaces that is component based. \par}

\medskip
{\noindent Option 2 - Aurelia \cite{Aurelia} \par}
{\noindent Aurelia is a newer JavaScript client framework for mobile, desktop, and the web, by using simplistic integration.  \par}

\medskip
{\noindent Option 3 - Ember \cite{Ember} \par}
{\noindent Ember uses web components and templates to increase productivity.  \par}

\medskip
{\noindent Option 4 - Angular2 \cite{Angular2} \par}
{\noindent Angular2 is a project started by Google for their internal Green Tea project. Angular2 is a widely documented JavaScript cross-platform library that is used to create native mobile and desktop web applications. \par}

\medskip
{\noindent\rmfamily\bfseries\color{black} Goals \par}
{\noindent The use of this technology will aid in the development of the user interface. Having a beautiful and scalable user interface will help users interact with the platform more easily, on whatever device they choose to use it on. \par}

\medskip
\newpage
{\noindent\rmfamily\bfseries\color{black} Evaluation Criteria \par}
{\noindent The options are evaluated on

\begin{itemize}
\item Ease of Use
\item File Size
\item Features
\item Performance
\item Standards Compliance
\item Non-Compliance
\item Release
\item License
\end{itemize}

 \par}

%\begin{center}

\newpage
{\noindent\rmfamily\bfseries\color{black} Option Comparison \par}
\vspace{1pc}
\tablehead{}
\begin{supertabular}{|p{2cm}|p{2.5cm}|p{2.5cm}|p{2.5cm}|p{2.5cm}|}
\hline \cite{Eisenberg} & React & Aurelia  & Ember  & Angular 2 \\ \hline
Ease of Use & Substantial setup required for working system, lots of documentation and tutorials. & Simple setup using NPM and installation     & Simple setup using NPM and installation & Substantial setup required for working system, lots of documentation and tutorials. \\ \hline
File Size & 156kb to ???kb, due to added frameworks & 323kb & 435kb                                   & 1023kb \\ \hline
Features & View rendering engine with plugin frameworks & Router, Animation, HTTP Client & Router, HTTP Client & Router, HTTP Client                                                                 \\ \hline
Performance (Paints per Second) & 45-50 & 90-150 (Higher end with additional plugins) & 60-100 & 80-130 (Higher end with additional plugins) \\ \hline
Standards Compliance & ES 2015 & HTML, ES 2016, Web Components & HTML, ES 2015                           & ES 2016\\ \hline
Non-Compliance & JSX                                                                                 & N/A & N/A & NG2 Markup, Dart \\ \hline
Release & 15.x & Beta & 2.x  & Release Candidate \\ \hline
License & BSD & MIT & MIT & MIT \\ \hline
\end{supertabular}

\newpage
{\noindent\rmfamily\bfseries\color{black} Discussion \par}
{\noindent All of the chosen options have their pros and cons for our web application. All of them however would be a learning and research experience. Angular2 and React have the benefit of being created by large software companies, Google and Facebook respectively. This means that there will be large adoption and documentation / tutorials available. Aurelia and Ember seem to be easier to implement however, they are much newer products and they have a smaller adoption population. This could prove troublesome if we run into problems. If our implementation ends up being a fork of Ward Cunningham's Federated Wiki, then this decision will be null most likely. \par}

\medskip
{\noindent\rmfamily\bfseries\color{black} Selection \par}
{\noindent Initially we were set on using Angular2 as part of the team has experience using this JavaScript library, before meeting with our client. Angular2 has a wide adoption and is used by Google for internal projects so the longevity of the framework is expected to last. With this in mind, we plan to use Angular2 if we need to use a JavaScript framework for our user interface. \par}



%\end{center}
\begin{comment}
\newpage
\subsubsection{Documentation}
{\noindent Option 1 - Microsoft Word \cite{Word} \par}
{\noindent  Microsoft Word is a fairly powerful and widespread software that is often used for documentation outside of higher education and the technology industry. \par}

\medskip
{\noindent Option 2 - Dozuki \cite{Dozuki} \par}
{\noindent Integrates revisions, approval, and documentation into a single system that allows for efficient work-flows for creating customizable and centrally managed documentation. \par}

\medskip
{\noindent Option 3 - LaTeX \cite{Latex} \par}
{\noindent  LaTeX, hereby known as Latex, is a document preparation system often used in the technical and scientific fields for documentation, and is the de facto standard for communication and publication of scientific documents \cite{Latex}. \par}

\medskip
{\noindent\rmfamily\bfseries\color{black} Goals \par}
{\noindent An efficient documentation solution that is able to provide the ability to create beautiful and useful documentation of  the platform. This documentation will assist the end user with operation of the various parts of the platform.\par}

\medskip
\newpage
{\noindent\rmfamily\bfseries\color{black} Evaluation Criteria \par}
{\noindent The options are evaluated on

\begin{itemize}
\item Ease of Use
\item File Size
\item Features
\item Performance
\item Cost
\end{itemize}

 \par}

%\begin{center}

\newpage
{\noindent\rmfamily\bfseries\color{black} Option Comparison \par}
\vspace{1pc}
\tablehead{}
\begin{supertabular}{|p{2cm}|p{4cm}|p{4cm}|p{4cm}|}
\hline  & Microsoft Word & Dozuki & LaTeX\\ \hline
Ease of Use & Microsoft Word is easy to use for most cases, but can be difficult to use when attempting certain formatting requests, such as creating a table of contents. & Dozuki is a newer documentation solution that spawned from iFixit, Dozuki prides itself on being easy to learn and use, while providing centralized   & Latex is able to take a lot of the tediousness one would find in Microsoft Word, allowing the user to focus on the document contents instead of appearance. "Steep learning curve, but not very high" \cite{McGrath} \\ \hline
File Size & Small to Large, depending on embedded images or graphics & Unknown & Often very small due to only text \\ \hline
Features & Moderately customizable, and can be 'pretty' & Highly customizable with CSS and HTML & Highly customizable and 'pretty' \\ \hline
Performance & Can become bogged down by long documents & Web interface limitation, exact performance unknown & Easily handles complex and length documents  \\ \hline
Cost & \$149 for Office Home \& Student 2016 & \$299 / month for 10 users, \$10 Additional Per User & Free \\ \hline

\end{supertabular}

\newpage
{\noindent\rmfamily\bfseries\color{black} Discussion \par}
{\noindent While we are highly experienced using Microsoft Word, the documents become bloated, and sometimes the software is prone to crashes when dealing with very large documents. Dozuki seems like a very nice documentation solution, that is used by Intel and other technology companies. Having a centralized management system allows management to verify everyone is looking at the most updated documentation. This would be a very nice solution, though the cost is very prohibitive, at over \$3500 / year. Latex is free, and is able to create highly structured documentation that is also easy to style. Latex is used throughout the science field for ease of use.\par}

\medskip
{\noindent\rmfamily\bfseries\color{black} Selection \par}
{\noindent Latex seems to be the best solution of the three, due to current experience using the software, as well as ease of use when creating large documents with multiple sections and headings. \par}

\end{comment}

\newpage
\subsubsection{User Login \& Authentication}
{\noindent Option 1 - OpenID Connect \par}
{\noindent OpenID Connect allows for clients of all types, including browser-based JavaScript and native mobile apps, to launch sign-in flows and receive verifiable assertions about the identity of signed-in users \cite{OpenID}. \par}

\medskip
{\noindent Option 2 - Facebook  \par}
{\noindent Facebook Login for Apps is a fast and convenient way for people to create accounts and log into your app across multiple platforms \cite{Facebook}. \par}

\medskip
{\noindent Option 3 - PHP \& SQL\par}
{\noindent Using PHP and SQL to compare submitted usernames and passwords against stored data on a database.\par}

\medskip
{\noindent\rmfamily\bfseries\color{black} Goals \par}
{\noindent An efficient and secure method for allowing for users to login and continue editing their documents from any computer or device they choose.\par}

\medskip
%\newpage
{\noindent\rmfamily\bfseries\color{black} Evaluation Criteria \par}
{\noindent The options are evaluated on

\begin{itemize}
\item Ease of Use
\item Features
\item Security

\end{itemize}

 \par}

\vspace{2pc}
%\begin{center}

\newpage
{\noindent\rmfamily\bfseries\color{black} Option Comparison \par}
\vspace{1pc}
\tablehead{}
\begin{supertabular}{|p{2cm}|p{4cm}|p{4cm}|p{4cm}|}
\hline  & OpenID & Facebook Login & PHP \& SQL\\ \hline
Ease of Use & Requires Credentials with Corresponding Login Providers, Lots of available libraries  & Requires Credentials with Facebook and an App ID with Facebook & Easy to implement, but if setup incorrectly can lead to problems\\ \hline
Security & Relies on credential host and user security & Relies on Facebook and user security & Relies on password protection implementation and user security\\ \hline
Features & Easily log in with OpenID partner credential hosts (Google, Microsoft, Yahoo, etc) & Easily log in with Facebook credentials & Easily log in with user created account and password \\ \hline

\end{supertabular}

\newpage
{\noindent\rmfamily\bfseries\color{black} Discussion \par}
{\noindent The ideal user authentication system would be a combination of all three of the above implementations. While logging in with Facebook would make it easier to determine who is using the service, preventing unauthorized users from accessing unreleased copyrighted material, not everyone has a Facebook. Additionally using an OpenID login system would be reliant on other platform holders that use OAuth 2.0. Finally, using a self created account is often the easiest and can allow users to not be tied to a given account and also prevent private information from being retrieved from user accounts.\par}

\medskip
{\noindent\rmfamily\bfseries\color{black} Selection \par}
{\noindent For our implementation, we plan on using initially a PHP and SQL system to validate user account information on our database. Additionally, we will look into adding both OpenID and Facebook Login down the road. \par}

\newpage
\subsubsection{Interface Design}
{\noindent Option 1 - User Centered Design \par}
{\noindent A deep understanding of the target audience is able to provide insights into how to design and develop your application to suit your intended users \cite{Usability}. \par}

\medskip
{\noindent Option 2 - Activity-Centered Design  \par}
{\noindent Instead of focusing on research about intended users, the design and development are focused around making a given activity logically designed \cite{AListApart}. \par}

\medskip
{\noindent Option 3 - Self Design \par}
{\noindent The designer is responsible for representing the target audience. Though this can be a poor representation of the intended audience \cite{AListApart}.\par}

\medskip
{\noindent\rmfamily\bfseries\color{black} Goals \par}
{\noindent A design principle that allows for user interfaces that lead to user interfaces that are accepted by users and are easy to understand.\par}

\medskip
%\newpage
{\noindent\rmfamily\bfseries\color{black} Evaluation Criteria \par}
{\noindent The options are evaluated on

\begin{itemize}
\item Ease of Use
\item Strengths
\item Weaknesses

\end{itemize}

\par}

\vspace{2pc}
%\begin{center}

\newpage
{\noindent\rmfamily\bfseries\color{black} Option Comparison \par}
\vspace{1pc}
\tablehead{}
\begin{supertabular}{|p{2cm}|p{4cm}|p{4cm}|p{4cm}|}
\hline  & User Centered Design & Activity-Centered Design & Self Design\\ \hline
Ease of Use & Long process, that takes a lot of data gathering to provide insights into a target audience. & Easier to design an activity when not trying to cater a specific audience. & Very easy to design what works well for you as a designer. \\ \hline
Strengths & Allows for the designer to understand what makes a user think the way they do. This allows for an interface design to be molded to an expected user. & Allows the designer to design a user interface based around an activity that a user will be performing instead of designing to a users wants and desires. & Allows for easy creation of user interface of how the designers see fit. Perfect for a target audience that is just like the designer. \\ \hline
Weaknesses & Takes a long time to gather enough information to be able to design a good solution that feels natural to a target audience user. & Designed interface might work well for an intended activity, but could be antagonistic to a target audience. & Interface could be intended for an entirely different audience, leaving a confusing experience.\\ \hline

\end{supertabular}

\newpage
{\noindent\rmfamily\bfseries\color{black} Discussion \par}
{\noindent The MVP Platform is highly user focused, which initially led the team to decided on User Centered Design. Activity-Centered Design or Self Design would greatly reduce the burden of research and discovery into what our target audience would like to see or be comfortable with naturally. Activity-Centered Design, if performed properly would result in interfaces that clearly work as intended, though might be off putting to our users. Self Design would allow for one of the team members to decide how a certain element shall look, but again can fall into an interface that does not satisfy our users.\par}

\medskip
{\noindent\rmfamily\bfseries\color{black} Selection \par}
{\noindent For our implementation, we plan on using User Centered Design. This is because users are our very important for our project. If our users do not like our user interface, then they will be less likely to use our software. \par}








%\vspace{1pc}
\newpage
\subsection{Josh}
\vspace{1pc}

\subsubsection{Testing}
{\noindent Option 1 - Mocha \par}
{\noindent Mocha is a JavaScript testing framework, loaded with features. It runs on Node.js and also in the browser, making asynchronous testing simple and easy to use. Mocha tests run serially, allowing for flexible and accurate reporting, while mapping uncaught exceptions to the correct test cases. \cite{Mocha} \par}

\medskip
{\noindent Option 2 - QUnit \par}
{\noindent QUnit is a powerful, easy-to-use JavaScript unit testing framework. It's used by the jQuery, jQuery UI and jQuery Mobile projects and is capable of testing any generic JavaScript code. \cite{QUnit}  \par}

\medskip
{\noindent Option 3 - Jasmine \par}
{\noindent Jasmine is a behavior-driven development framework for testing JavaScript code. It does not depend on any other JavaScript frameworks. It does not require a DOM. And it has a clean, obvious syntax so that you can easily write tests. \cite{Jasmine}  \par}

\medskip
{\noindent\rmfamily\bfseries\color{black} Goals \par}
{\noindent Using this technology will aid in proper functionality and minimize errors. Without properly testing code, a number of problems can occur that can disrupt and slow down progress in a team. In extreme cases, not properly testing could lead to failure of the application. \par}

\medskip
\newpage
{\noindent\rmfamily\bfseries\color{black} Evaluation Criteria \par}
{\noindent The options are evaluated on

\begin{itemize}
\item Ease of Use
\item Features
\item Documentation
\item Integration
\item Other
\end{itemize}

 \par}

%\begin{center}

\newpage
{\noindent\rmfamily\bfseries\color{black} Option Comparison \par}
\tablehead{}
\begin{supertabular}{|p{2cm}|p{2.5cm}|p{2.5cm}|p{2.5cm}|p{2.5cm}|}
\hline & Mocha & QUnit & Jasmine\\ \hline
Language is like spoken language, which makes for easier to understand & Not as friendly, language is similiar to other basic Unit testing frameworks & Language is like spoken English, fairly easy to understand\\ \hline
Rich in features: Spys, mocks, stubs, callbacks, etc. Most features & Most testing is done through assertions only, not as many features & Rich in features: Spys, mocks, stubs, callbacks, etc. Lacks Assertions, but can be implemented with library\\ \hline
Fairly common, moderate amount of documentation & Not well documented & Most popular, abundant documentation \\ \hline
Moderate difficulty to set up & Easiest to set up & Moderate difficulty to set up\\ \hline
&  & Familiarity\\ \hline
\end{supertabular}

\newpage
{\noindent\rmfamily\bfseries\color{black} Discussion \par}
{\noindent It is well known that an application that ensures proper functionality supersedes other applications and, more importantly, competitors. Quality assurance is acknowledged as highly distinguished, and therefore, an attribute deserving of notable consideration. Having this in mind, we will be considering these mentioned testing frameworks: Mocha, QUnit, and Jasmine. \par}

\medskip
{\noindent\rmfamily\bfseries\color{black} Selection \par}
{\noindent Jasmine, even though more difficult to integrate and set up, would be most feasible. Jasmine has many features that win out over Mocha and QUnit, as well as thorough documenation. The test are easily grouped through describe blocks, which makes it easy to identify where certain problems are. There is familiarity with Jasmine over the other testing frameworks, which means less time learning for the team.\par}

%\end{center}

\medskip

\newpage
\subsubsection{Version Control}
{\noindent Further information is needed to begin planning built in source control. \par}

{\noindent Option 1 - Git \par}
{\noindent \par}

\medskip
{\noindent Option 2 - \par}
{\noindent \par}

\medskip
{\noindent Option 3 - \par}
{\noindent \par}

\medskip
{\noindent\rmfamily\bfseries\color{black} Goals \par}
{\noindent \par}

\medskip
\newpage
{\noindent\rmfamily\bfseries\color{black} Evaluation Criteria \par}
{\noindent The options are evaluated on 

\begin{itemize}
\item API
\item License
\item Integration
\item Other
\end{itemize}

 \par}

%\begin{center}

\newpage
{\noindent\rmfamily\bfseries\color{black} Option Comparison \par}
\tablehead{}
\begin{supertabular}{|p{2cm}|p{2.5cm}|p{2.5cm}|p{2.5cm}|p{2.5cm}|}
\hline & Option 1 & Option 2 & Option 3\\ \hline
\end{supertabular}

\newpage
{\noindent\rmfamily\bfseries\color{black} Discussion \par}
{\noindent  \par}

\medskip
{\noindent\rmfamily\bfseries\color{black} Selection \par}
{\noindent \par}

\medskip

\newpage
\subsubsection{Database}
{\noindent Option 1 - SQL Server \par}
{\noindent SQL Server is owned by Microsoft, and is known for its scalable, hybrid database platform. Being owned by Microsoft, this database is well known. \cite{SQL Server} \par}

\medskip
{\noindent Option 2 - MongoDB \par}
{\noindent MongoDB is a document based database, with an Expressive Query Language and Secondary Indexes. \cite{MongoDB}  \par}

\medskip
{\noindent Option 3 - MySQL \par}
{\noindent Many of the worlds largest companies rely on MySQL, such as Facebook, Google, and others. It is a relational database. \cite{MySQL}  \par}

\medskip
{\noindent\rmfamily\bfseries\color{black} Goals \par}
{\noindent Having an understandable, well built database can aid in the flow of building an application as well as delivery. This in mind, we want to balance cost with effectiveness. \par}

\medskip
\newpage
{\noindent\rmfamily\bfseries\color{black} Evaluation Criteria \par}
{\noindent The options are evaluated on 

\begin{itemize}
\item Cost
\item Type
\item Ease of Use
\end{itemize}

 \par}

%\begin{center}

\newpage
{\noindent\rmfamily\bfseries\color{black} Option Comparison \par}
\tablehead{}
\begin{supertabular}{|p{2cm}|p{2.5cm}|p{2.5cm}|p{2.5cm}|p{2.5cm}|}
\hline & SQL Server & MongoDB & MySQL\\ \hline
Enterprise: \$12,256, Standard: \$3717, Developer: Free & OpenSource, but essential periphery software can range from \$45 a month to as much as \$5225 a month & Enterprise: \$5000, Standard: \$2000, Community Edition: Free \\ \hline
\hline Structured Query Language & Expressive Query Language & Structured Query Language\\ \hline
\hline Uses Queries to gather information & EQL data access and manipulation in sophisticated ways, supporting operational and analytical applications \cite{ibmbpnetwork} & Uses Queries to gather information \\ \hline
\end{supertabular}

\newpage
{\noindent\rmfamily\bfseries\color{black} Discussion \par}
{\noindent The most obvious and transparent qualities for a database are the cost and type, however, databases that come with a free edition allow for us to test and experiment while developing. This is imperative to the development process, and shouldn't be overlooked. Allowing for an easy transition from the free edition to a standard or express edition is important as well. A non relational database like MongoDB could be beneficial when not knowing the type of data that will be stored.' \par}

\medskip
{\noindent\rmfamily\bfseries\color{black} Selection \par}
{\noindent All things considered, SQL Server would be considered our top pick. Experience using an SQL based database before aids in the difficulty of using a query. Having a free edition contributes a moderate amount into the decision as well.\par}

\medskip








%\vspace{1pc}
\newpage
\subsection{Evan}
\vspace{1pc}
\subsubsection{Server Back-end}
{\noindent Option 1 - NodeJS \par}
{\noindent NodeJS is a modern web back-end framework developed by the Node Foundation,
primarily led by Joyent. By using JavaScript its language of choice, Node allows
developers to use the language's unique concurrency paradigms to quickly develop
scalable applications. \par}
{\noindent Option 2 - Django \par}
{\noindent The Django framework is a massive web framework developed in Python that comes "batteries
included". The Framework includes everything from geo-libraries to support for four different kinds of
databases, meaning a large initial learning curve but a large payoff. \par}
{\noindent Option 3 - Flask \par}
{\noindent Flask is a micro-framework. It comes with the bare minimum needed to do HTTP handling, leaving
what other frameworks come with to an array of choices from third party developers. This means the core
framework is quick to learn, but can quickly leave a developer feeling constrained. \par}
{\noindent Option 4 - Ruby on Rails \par}
{\noindent Ruby on Rails is the old standard of web frameworks. It was the original batteries included
framework, and has over the years been known for its ease of use. However, the framework is quite old
and shows some signs of age, using sometimes outdated paradigms and generally being less friendly to beginners
than more modern frameworks. \par}

\medskip
\newpage
{\noindent\rmfamily\bfseries\color{black} Evaluation Criteria \par}
{\noindent The options are evaluated on

\begin{itemize}
\item Ease of use
\item Language
\item Features
\item Ecosystem
\item License
\end{itemize}

 \par}

%\begin{center}

\newpage
{\noindent\rmfamily\bfseries\color{black} Option Comparison \par}
\vspace{1pc}
\tablehead{}
\begin{supertabular}{|p{2cm}|p{2.5cm}|p{2.5cm}|p{2.5cm}|p{2.5cm}|}
\hline & NodeJS & Django  & Flask  & Ruby on Rails \\ \hline
Ease of Use
  & Javascript backend shares language with frontend; super quick iteration
  & Large framework containing all needs within, with a large learning curve
  & Microframework containing only minimal needs; requires finding external packages
  & Old framework with massive, but aging, ecosystem \\ \hline
Language
  & Javascript
  & Python
  & Python
  & Ruby \\ \hline
Features
  & Minimal HTTP interaction, massive external ecosystem for extras
  & Massive built-ins
  & Minimal with external ecosystem for extras
  & Massive built-ins \\ \hline
Ecosystem
  & Massively popular today with expansive and growing ecosystem
  & Decently large ecosystem with built-ins for most tasks
  & Decently large ecosystem
  & Large ecosystem but fairly old \\ \hline
Release
  & 7.1.0
  & 1.10.3
  & 0.11
  & 5.0.0.1 \\ \hline
License
  & MIT
  & BSD
  & BSD
  & MIT \\ \hline
\end{supertabular}

\newpage
{\noindent\rmfamily\bfseries\color{black} Discussion \par}
{\noindent
All backends listed are popular within their respective communities. However, more new projects are being created using NodeJS, as its modern paradigms and sharing of a language with frontend development allow developers familiar with it to iterate quicker and write more expressive code. Django's built-ins allow for a quicker initial development time but mean being isolated from the rapidly expanding ecosystem around NodeJS.
Ruby on Rails is a rather old backend, and has not shown to have the modern flexibility of Node.
 \par}

\medskip
{\noindent\rmfamily\bfseries\color{black} Selection \par}
{\noindent As NodeJS has the most expansive ecosystem, and allows us to share a common language between the front and backennds of the project, we will be using it over the other options considered. Additionally, Ward Cunningham's Federated Wiki uses it as one of its backends, and if we fork it, we can continue to use its NodeJS backend. \par}

\newpage
\subsubsection{Text Formatting}
{\noindent \par}
{\noindent Option 1 - Markdown \par}
{\noindent Markdown is a highly lightweight markup language that allows easy, human-readable markup
of text to include headings, bold/italic/underline/etc, bullets, and numbered lists. The original markdown does not support
things like images or videos; Markdown has various "flavors", or implementations, that sometimes
allow for such things. \par}
{\noindent Option 2 - Restructured Text \par}
{\noindent Restructured Text is a markup language written in Python for writing documentation, simple websites,
etc. It allows for highly varied but still restricted markup; it allows for image embeds, fancy linking,
titles, etc. It does not allow users to embed arbitrary elements. \par}
{\noindent Option 3 - Raw HTML \par}
{\noindent Storing simply raw HTML allows the greatest flexibility, as it is literally the same elements
rendered in browser. Raw HTML allows for things like scripting, video embeds, etc., and as such must
be filtered to a restricted subset to be suitable for use in a public-facing scenario. \par}

\medskip
\newpage
{\noindent\rmfamily\bfseries\color{black} Evaluation Criteria \par}
{\noindent The options are evaluated on

\begin{itemize}
\item Ease of use for end-users
\item Markup options
\item Compile language
\item Security
\end{itemize}

 \par}

%\begin{center}

{\noindent\rmfamily\bfseries\color{black} Option Comparison \par}
\vspace{1pc}
\tablehead{}
\begin{supertabular}{|p{2cm}|p{2.5cm}|p{2.5cm}|p{2.5cm}|p{2.5cm}|}
\hline
  & Markdown
  & ReStructured Text
  & HTML \\ \hline
Ease of Use
  & Easy to use, with minimal options and human-readable markup; different implementations have slight differences leading to confusion
  & Relatively human-readable markup but with massive number of options
  & Essentially infinite options but not very human-readable/human-writeable  \\ \hline
Markup
  & Options readable in single-page document, not allowing for high flexibility
  & Highly featureful with well-defined language
  & Allows for infinite options along; can use CSS to fine-tune display \\ \hline
Compile Language
  & Dozens of different libraries for different languages, each with slightly different interpretation of the markup
  & Python
  & No compilation needed to display but some backend processing needed for security \\ \hline
Security
  & Small language leads to minimal exploits
  & Well-defined with real-world-tested libraries
  & Needs careful processing to stop end-users from inputting harmful raw input (including scripts)
  \\ \hline
\end{supertabular}

{\noindent\rmfamily\bfseries\color{black} Discussion \par}
{\noindent
All languages listed allow users to do simple things like bold text and link to other pages. However,
HTML offers the most flexibility and allows users to be able to do anything they want.
Allowing this while maintaining ease-of-use would require a frontend library
that can allow users to interact with the document in "what you see is what you get",
or WYSIWYG, mode.
 \par}

\medskip
{\noindent\rmfamily\bfseries\color{black} Selection \par}
{\noindent
The Federated Wiki uses HTML with minimal security processing. If we fork the Federated Wiki, we will use
HTML with some added processing to increase its security. Otherwise, for ease of implementation, we will
use Markdown for its backend language support, but implemented in such a way as to allow easy
replacement of the code with some other markup language as wanted.
\par}

\newpage
\medskip
\subsubsection{Password Storage}
{\noindent \par}
{\noindent Option 1 - Bcrypt \par}
{\noindent Bcrypt is a password hashing function that takes a very large amount of time to crack
an individual password -- it is designed to be slow. This means a hacker cannot simply crack
a database worth of passwords in one sitting, as with older hashes like MD5. \par}
{\noindent Option 2 - Scrypt \par}
{\noindent Scrypt is designed to take up large amounts of time, and large amounts of RAM,
when hashing. This ensures that a hacker cannot simply buy a powerful CPU and crack passwords with
pure power. However, scrypt, being designed more-so for computer hard disk passwords, can take
multiple seconds and hundreds of megabytes of RAM to process. \par}
{\noindent Option 3 - pbkdf2 \par}
{\noindent PBKDF2 is a function that repeatedly hashes a password using the HMAC, or "keyed-hash
message authentication code", function. For a CPU, cracking a large number of passwords using pbkdf2
is difficult, as it takes a large amount of time to crack an individual password. Using a GPU,
however, a large number of hashes can be run in parallel, making it quick to crack with high end hardware. \par}
{\noindent Option 4 - raw storage \par}
{\noindent Another option for password storage is to store the passwords in plain text. This allows
users to recover their passwords directly through a password reminder email. However, this comes with
the major downside that compromising the database allows a hacker to be able to access any
accounts on unrelated services where users use the same username and password (a common pattern
in non-technical and technical users alike). \par}

\medskip
\newpage
{\noindent\rmfamily\bfseries\color{black} Evaluation Criteria \par}
{\noindent The options are evaluated on

\begin{itemize}
\item Cracking
\item Storage
\item Resistance to Hacking
\end{itemize}

 \par}

%\begin{center}

\newpage
{\noindent\rmfamily\bfseries\color{black} Option Comparison \par}
\vspace{1pc}
\tablehead{}
\begin{supertabular}{|p{2cm}|p{2.5cm}|p{2.5cm}|p{2.5cm}|p{2.5cm}|}
\hline
  & Bcrypt
  & Scrypt
  & pbkdf2
  & plain text \\ \hline
Cracking
  & Bcrypt is highly resistant to cracking on CPUs and GPUs, but can be cracked quickly using specialized FPGAs.
  & Scrypt is highly resistant to cracking on CPUs, GPUs, etc but takes a large amount of time to verify a valid password.
  & pbkdf2 is highly resistant to cracking on CPUs but can be easily cracked on a GPU.
  & A plaintext password does not need to be cracked as it is already stored as a raw password. \\ \hline
Storage
  & can be stored in a database without issue.
  & can be stored in a database without issue.
  & can be stored in a database without issue.
  & Plain passwords must be stored in a way that ensures they can never be hacked, which is impossible. \\ \hline
Resistance to Hacking
  & A bcrypt password hash must be cracked before it can be used elsewhere.
  & Scrypt hashes must be cracked before use elsewhere.
  & pbkdf2 hashes must be cracked before use elsewhere.
  & A plain password, once retrieved from a database, can be used along with the username/email associated to hack other sites. \\ \hline
\end{supertabular}

{\noindent\rmfamily\bfseries\color{black} Discussion \par}
{\noindent
Password storage is a tradeoff between ease of use and difficulty of reversing. Scrypt is too slow for use on a website with many users,
whereas plain text passwords are too insecure as a hacker can reuse the password immediately without cracking. pbkdf2 and bcrypt do a good
job defending against CPU cracking, but as pbkdf2 can be cracked using a GPU, bcrypt is left remaining as the best tradeoff between speed and
cracking.
 \par}

\medskip
{\noindent\rmfamily\bfseries\color{black} Selection \par}
{\noindent
As mentioned above, bcrypt-hashed passwords present a good tradeoff between cracking ability and verification speed.
As such, the Many Voices Platform will use bcrypt to securely verify any password used with the system.
\par}

\newpage
\subsection{Conclusion}
{\noindent  The Many Voices Publishing Platform is a combination of User Interfaces, Documentation, User Centered Design, Testing, User Authentication, Databases, Server Back-end, Text Formatting, Password Storage, and the users themselves. Determining the technologies behind these parts and pieces is a difficult task to accomplish, as many choices can satisfy the requirements of the project. Finding the best solution however is the goal of this document, to provide a clear path forward for the platform as a whole. \par}

\newpage
\bibliographystyle{IEEEtran}
\bibliography{techreview}


\end{document}
