\documentclass[onecolumn, draftclsnofoot,10pt, compsoc]{IEEEtran}

\usepackage{supertabular}

\usepackage{fancyhdr}

\pagestyle{fancy}
\fancyfoot[C]{DESIGNCHANGE--\thepage}
\begin{document}
\section{What Design Decisions Changed?}

Throughout development of the Many Voices Publishing Platform, only a few changes were made throughout development that strayed from the original requirements and planning.

\begin{flushleft}
	\tablehead{}
	\begin{supertabular}{|p{0.5cm}|p{3cm}|p{5.5cm}|p{7cm}|}
		\hline
		 & Design & Description of What Happened & Comments
		\\\hline
		1 & User login \& Authentication & We were unsure if we would be using 3rd party login or rolling our own; we used Google's oAuth backend & 3rd party login allowed us to simplify the new account creation experience.
		\\\hline
		2 & Book compilation & LaTeX files are constructed in-memory and saved to disk before compilation, rather than being saved individually and constructed with \verb|\include| and \verb|\input|. & This allows us to more easily change the document design and to implement an easier data flow.
		\\\hline
		3 & Make text formatting transparent to user & Users can choose to write scraps with full LaTeX enabled, or in plain-text mode. & This change allows for much more powerful document creation. Creating a table or a chart is now possible, which would not have been possible if the user had to use a GUI-based editor.
		\\\hline
		4 & Use passwords to authenticate users & Instead of passwords, a user logs in with their Google account. A token is generated that is then stored in their browser and in the backend database. & Users no longer need to remember a password and instead of storing a password, we can store unique, randomly generated tokens, removing the risk of password leakage.
		\\\hline
		5 & Passwords must be salted and hashed & Instead of passwords, tokens are used. Tokens are hashed but not salted to allow for token lookup directly. & Tokens do not need to be salted, as they are already unique to the service and leaking a token does not harm user security. Tokens are stored salted, and so still require a large amount of computation to break a single token.
		\\\hline
		6 & Use a SQL database for user data & Instead of a SQL database, a document storage database was used. & Using a document storage database made sense, since the data we were storing was already created as document-style JSON objects. Storing these and retrieving these as JSON simplified the marshalling to and from SQL statements would have required.
		\\ \hline
	\end{supertabular}
\end{flushleft}


\end{document}
