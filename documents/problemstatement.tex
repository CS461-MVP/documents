 \documentclass[letterpaper, 10pt, draftclsnofoot, onecolumn]{IEEEtran}
\begin{document}
    \setlength{\parindent}{0pt}
    \setlength{\parskip}{0pt}
\begingroup\flushleft
  Joshua Matteson\\
  Steven Powers\\
  Evan Tschuy
\endgroup

\setlength{\parindent}{4em}
\linespread{1.1}

\vspace{1pc}
\centerline{\sc \large Problem Statement}
\vspace{2pc}

The purpose of the Many Voices Publishing Platform project is to remedy problems 
associated with the current textbook market and standard expectations that come 
with textbooks. The Many Voices Publishing Platform was developed to alleviate 
costs of text-books for students and provide instructors the ability 
to create them own textbooks using an open platform for collaboration between 
content creators and instructors and specificity of focus within their course materials. 
This platform was created using Ward Cunningham’s federated wiki as a project 
base and was expanded upon to provide the ability to define and implement a 
collaborative platform for authoring.

Today the textbook industry is filled with yearly updates that are often released with the
minimum number of changes needed to justify the new edition. Additionally, the addition of 
online course materials are used to keep students and professors buying new textbooks. 
A professor can either choose a textbook from a catalog, or write one of their own. 
Pre-written textbooks often contain unnecessary chapters, or can even be missing chapters 
viewed as critical. Writing one's own textbook, on the other hand, can take months or 
years of research and editing.

The Many Voices platform instead allows professors to create their own textbooks 
by using crowd-sourced content. Content creators are able to submit materials of 
many varieties, including images, small pieces of text, pages, or even entire chapters. 
A knowledgeable professor can contribute a chapter or unit on their specialty, and 
others are able to modify obtained materials in the way they see fit. These materials may be
offered for free or for a nominal fee, determined by the content creator. Once materials are 
acquired, anyone is able to create a more affordable textbook available in either
a downloadable ebook or in print form to create the exact textbook they want for their classes
or for any purpose they desire. The finished platform will allow users to see existing 
textbooks, customize them using their own or third party materials, and contribute their 
customizations back to other professors.

To ensure the platform meets the defined needs and functions as according to plan, 
we will, in our best efforts, plan to meet at minimum every two weeks with our client to 
discuss current progress of project goals and proposed functionality for the application. 
Before beginning work on major milestones aspects of the platform, we will discuss our present 
status and inquire as to whether our current progress is in line to meet project expectations as
desired. After reaching an agreement, we will continue to adjust and shape the platform where 
necessary and reasonable.

\end{document}
