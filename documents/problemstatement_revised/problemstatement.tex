 \documentclass[letterpaper, 10pt, draftclsnofoot, onecolumn]{IEEEtran}

\usepackage{graphicx}                                        
\usepackage{amssymb}                                         
\usepackage{amsmath}                                         
\usepackage{amsthm}                                          

\usepackage{alltt}                                           
\usepackage{float}
\usepackage{color}
\usepackage{url}

\usepackage{balance}
\usepackage[TABBOTCAP, tight]{subfigure}
\usepackage{enumitem}
\usepackage{pstricks, pst-node}

\usepackage{hyperref}
\usepackage{geometry}

\usepackage{listings}
\usepackage{color}
\usepackage[doublespacing]{setspace}

%\setlength{\parindent}{4em}
%\linespread{1.1}


\def\name{Steven Powers, Josh Matteson, Evan Tschuy}

%pull in the necessary preamble matter for pygments output
%\usepackage{fancyvrb}
\usepackage{color}
\usepackage[latin1]{inputenc}


\makeatletter
\def\PY@reset{\let\PY@it=\relax \let\PY@bf=\relax%
    \let\PY@ul=\relax \let\PY@tc=\relax%
    \let\PY@bc=\relax \let\PY@ff=\relax}
\def\PY@tok#1{\csname PY@tok@#1\endcsname}
\def\PY@toks#1+{\ifx\relax#1\empty\else%
    \PY@tok{#1}\expandafter\PY@toks\fi}
\def\PY@do#1{\PY@bc{\PY@tc{\PY@ul{%
    \PY@it{\PY@bf{\PY@ff{#1}}}}}}}
\def\PY#1#2{\PY@reset\PY@toks#1+\relax+\PY@do{#2}}

\expandafter\def\csname PY@tok@gd\endcsname{\def\PY@tc##1{\textcolor[rgb]{0.63,0.00,0.00}{##1}}}
\expandafter\def\csname PY@tok@gu\endcsname{\let\PY@bf=\textbf\def\PY@tc##1{\textcolor[rgb]{0.50,0.00,0.50}{##1}}}
\expandafter\def\csname PY@tok@gt\endcsname{\def\PY@tc##1{\textcolor[rgb]{0.00,0.25,0.82}{##1}}}
\expandafter\def\csname PY@tok@gs\endcsname{\let\PY@bf=\textbf}
\expandafter\def\csname PY@tok@gr\endcsname{\def\PY@tc##1{\textcolor[rgb]{1.00,0.00,0.00}{##1}}}
\expandafter\def\csname PY@tok@cm\endcsname{\let\PY@it=\textit\def\PY@tc##1{\textcolor[rgb]{0.25,0.50,0.50}{##1}}}
\expandafter\def\csname PY@tok@vg\endcsname{\def\PY@tc##1{\textcolor[rgb]{0.10,0.09,0.49}{##1}}}
\expandafter\def\csname PY@tok@m\endcsname{\def\PY@tc##1{\textcolor[rgb]{0.40,0.40,0.40}{##1}}}
\expandafter\def\csname PY@tok@mh\endcsname{\def\PY@tc##1{\textcolor[rgb]{0.40,0.40,0.40}{##1}}}
\expandafter\def\csname PY@tok@go\endcsname{\def\PY@tc##1{\textcolor[rgb]{0.50,0.50,0.50}{##1}}}
\expandafter\def\csname PY@tok@ge\endcsname{\let\PY@it=\textit}
\expandafter\def\csname PY@tok@vc\endcsname{\def\PY@tc##1{\textcolor[rgb]{0.10,0.09,0.49}{##1}}}
\expandafter\def\csname PY@tok@il\endcsname{\def\PY@tc##1{\textcolor[rgb]{0.40,0.40,0.40}{##1}}}
\expandafter\def\csname PY@tok@cs\endcsname{\let\PY@it=\textit\def\PY@tc##1{\textcolor[rgb]{0.25,0.50,0.50}{##1}}}
\expandafter\def\csname PY@tok@cp\endcsname{\def\PY@tc##1{\textcolor[rgb]{0.74,0.48,0.00}{##1}}}
\expandafter\def\csname PY@tok@gi\endcsname{\def\PY@tc##1{\textcolor[rgb]{0.00,0.63,0.00}{##1}}}
\expandafter\def\csname PY@tok@gh\endcsname{\let\PY@bf=\textbf\def\PY@tc##1{\textcolor[rgb]{0.00,0.00,0.50}{##1}}}
\expandafter\def\csname PY@tok@ni\endcsname{\let\PY@bf=\textbf\def\PY@tc##1{\textcolor[rgb]{0.60,0.60,0.60}{##1}}}
\expandafter\def\csname PY@tok@nl\endcsname{\def\PY@tc##1{\textcolor[rgb]{0.63,0.63,0.00}{##1}}}
\expandafter\def\csname PY@tok@nn\endcsname{\let\PY@bf=\textbf\def\PY@tc##1{\textcolor[rgb]{0.00,0.00,1.00}{##1}}}
\expandafter\def\csname PY@tok@no\endcsname{\def\PY@tc##1{\textcolor[rgb]{0.53,0.00,0.00}{##1}}}
\expandafter\def\csname PY@tok@na\endcsname{\def\PY@tc##1{\textcolor[rgb]{0.49,0.56,0.16}{##1}}}
\expandafter\def\csname PY@tok@nb\endcsname{\def\PY@tc##1{\textcolor[rgb]{0.00,0.50,0.00}{##1}}}
\expandafter\def\csname PY@tok@nc\endcsname{\let\PY@bf=\textbf\def\PY@tc##1{\textcolor[rgb]{0.00,0.00,1.00}{##1}}}
\expandafter\def\csname PY@tok@nd\endcsname{\def\PY@tc##1{\textcolor[rgb]{0.67,0.13,1.00}{##1}}}
\expandafter\def\csname PY@tok@ne\endcsname{\let\PY@bf=\textbf\def\PY@tc##1{\textcolor[rgb]{0.82,0.25,0.23}{##1}}}
\expandafter\def\csname PY@tok@nf\endcsname{\def\PY@tc##1{\textcolor[rgb]{0.00,0.00,1.00}{##1}}}
\expandafter\def\csname PY@tok@si\endcsname{\let\PY@bf=\textbf\def\PY@tc##1{\textcolor[rgb]{0.73,0.40,0.53}{##1}}}
\expandafter\def\csname PY@tok@s2\endcsname{\def\PY@tc##1{\textcolor[rgb]{0.73,0.13,0.13}{##1}}}
\expandafter\def\csname PY@tok@vi\endcsname{\def\PY@tc##1{\textcolor[rgb]{0.10,0.09,0.49}{##1}}}
\expandafter\def\csname PY@tok@nt\endcsname{\let\PY@bf=\textbf\def\PY@tc##1{\textcolor[rgb]{0.00,0.50,0.00}{##1}}}
\expandafter\def\csname PY@tok@nv\endcsname{\def\PY@tc##1{\textcolor[rgb]{0.10,0.09,0.49}{##1}}}
\expandafter\def\csname PY@tok@s1\endcsname{\def\PY@tc##1{\textcolor[rgb]{0.73,0.13,0.13}{##1}}}
\expandafter\def\csname PY@tok@sh\endcsname{\def\PY@tc##1{\textcolor[rgb]{0.73,0.13,0.13}{##1}}}
\expandafter\def\csname PY@tok@sc\endcsname{\def\PY@tc##1{\textcolor[rgb]{0.73,0.13,0.13}{##1}}}
\expandafter\def\csname PY@tok@sx\endcsname{\def\PY@tc##1{\textcolor[rgb]{0.00,0.50,0.00}{##1}}}
\expandafter\def\csname PY@tok@bp\endcsname{\def\PY@tc##1{\textcolor[rgb]{0.00,0.50,0.00}{##1}}}
\expandafter\def\csname PY@tok@c1\endcsname{\let\PY@it=\textit\def\PY@tc##1{\textcolor[rgb]{0.25,0.50,0.50}{##1}}}
\expandafter\def\csname PY@tok@kc\endcsname{\let\PY@bf=\textbf\def\PY@tc##1{\textcolor[rgb]{0.00,0.50,0.00}{##1}}}
\expandafter\def\csname PY@tok@c\endcsname{\let\PY@it=\textit\def\PY@tc##1{\textcolor[rgb]{0.25,0.50,0.50}{##1}}}
\expandafter\def\csname PY@tok@mf\endcsname{\def\PY@tc##1{\textcolor[rgb]{0.40,0.40,0.40}{##1}}}
\expandafter\def\csname PY@tok@err\endcsname{\def\PY@bc##1{\setlength{\fboxsep}{0pt}\fcolorbox[rgb]{1.00,0.00,0.00}{1,1,1}{\strut ##1}}}
\expandafter\def\csname PY@tok@kd\endcsname{\let\PY@bf=\textbf\def\PY@tc##1{\textcolor[rgb]{0.00,0.50,0.00}{##1}}}
\expandafter\def\csname PY@tok@ss\endcsname{\def\PY@tc##1{\textcolor[rgb]{0.10,0.09,0.49}{##1}}}
\expandafter\def\csname PY@tok@sr\endcsname{\def\PY@tc##1{\textcolor[rgb]{0.73,0.40,0.53}{##1}}}
\expandafter\def\csname PY@tok@mo\endcsname{\def\PY@tc##1{\textcolor[rgb]{0.40,0.40,0.40}{##1}}}
\expandafter\def\csname PY@tok@kn\endcsname{\let\PY@bf=\textbf\def\PY@tc##1{\textcolor[rgb]{0.00,0.50,0.00}{##1}}}
\expandafter\def\csname PY@tok@mi\endcsname{\def\PY@tc##1{\textcolor[rgb]{0.40,0.40,0.40}{##1}}}
\expandafter\def\csname PY@tok@gp\endcsname{\let\PY@bf=\textbf\def\PY@tc##1{\textcolor[rgb]{0.00,0.00,0.50}{##1}}}
\expandafter\def\csname PY@tok@o\endcsname{\def\PY@tc##1{\textcolor[rgb]{0.40,0.40,0.40}{##1}}}
\expandafter\def\csname PY@tok@kr\endcsname{\let\PY@bf=\textbf\def\PY@tc##1{\textcolor[rgb]{0.00,0.50,0.00}{##1}}}
\expandafter\def\csname PY@tok@s\endcsname{\def\PY@tc##1{\textcolor[rgb]{0.73,0.13,0.13}{##1}}}
\expandafter\def\csname PY@tok@kp\endcsname{\def\PY@tc##1{\textcolor[rgb]{0.00,0.50,0.00}{##1}}}
\expandafter\def\csname PY@tok@w\endcsname{\def\PY@tc##1{\textcolor[rgb]{0.73,0.73,0.73}{##1}}}
\expandafter\def\csname PY@tok@kt\endcsname{\def\PY@tc##1{\textcolor[rgb]{0.69,0.00,0.25}{##1}}}
\expandafter\def\csname PY@tok@ow\endcsname{\let\PY@bf=\textbf\def\PY@tc##1{\textcolor[rgb]{0.67,0.13,1.00}{##1}}}
\expandafter\def\csname PY@tok@sb\endcsname{\def\PY@tc##1{\textcolor[rgb]{0.73,0.13,0.13}{##1}}}
\expandafter\def\csname PY@tok@k\endcsname{\let\PY@bf=\textbf\def\PY@tc##1{\textcolor[rgb]{0.00,0.50,0.00}{##1}}}
\expandafter\def\csname PY@tok@se\endcsname{\let\PY@bf=\textbf\def\PY@tc##1{\textcolor[rgb]{0.73,0.40,0.13}{##1}}}
\expandafter\def\csname PY@tok@sd\endcsname{\let\PY@it=\textit\def\PY@tc##1{\textcolor[rgb]{0.73,0.13,0.13}{##1}}}

\def\PYZbs{\char`\\}
\def\PYZus{\char`\_}
\def\PYZob{\char`\{}
\def\PYZcb{\char`\}}
\def\PYZca{\char`\^}
\def\PYZam{\char`\&}
\def\PYZlt{\char`\<}
\def\PYZgt{\char`\>}
\def\PYZsh{\char`\#}
\def\PYZpc{\char`\%}
\def\PYZdl{\char`\$}
\def\PYZti{\char`\~}
% for compatibility with earlier versions
\def\PYZat{@}
\def\PYZlb{[}
\def\PYZrb{]}
\makeatother


% The following metadata will show up in the PDF properties
\hypersetup{
  colorlinks = true,
  urlcolor = black,
  pdfauthor = {\name},
  pdfkeywords = {cs461 ``senior capstone'' problem statement},
  pdftitle = {CS 461 Problem Statement},
  pdfsubject = {CS 461 Problem Statement},
  pdfpagemode = UseNone
}



\begin{document}


\begin{titlepage}
\centering
{\huge Many Voices Publishing Platform\par}
{\LARGE Problem Statement\par}
{\vspace{2mm}}
{\large D. Kevin McGrath \& Dr. Kirsten Winters -  CS461 Fall 2016\par}
{\large Steven Powers, Josh Matteson, Evan Tschuy\par}
{\vspace{10mm}}

\abstract{\noindent The purpose of the Many Voices Publishing Platform 
project is to remedy problems associated with the current textbook market 
and standard expectations that come with textbooks. 
The Many Voices Publishing Platform was developed to alleviate 
costs of textbooks for students and provide instructors with the ability 
to create their own. Instructors can use the open platform for 
collaboration between content creators, and other instructors in order 
to choose the specific focus of their course materials. 
This platform was created using Ward Cunningham's federated 
wiki as a project base and was expanded upon to provide 
the ability to implement a collaborative platform for authoring.}

\end{titlepage}

\vspace{1pc}
\section{Problem Statement}
\vspace{2pc}

\vspace{1pc}
\subsection{Problem Definition}
\vspace{1pc}

{\noindent Today, the textbook industry is filled with yearly updates that 
are often released with the minimum number of changes needed to 
justify a new edition. These textbooks are repeatedly in the range of 
hundreds of dollars and are often filled with poor or even incorrect 
information. 
In the United States, most textbook materials are decided by a few 
states that choose what will appear in textbooks published nation wide. 
Most professors will simply choose the book provided to them for 
review that satisfies their requirements the best, without considering 
the costs that will be passed onto students. \par}

\medskip

{\noindent Furthermore, the addition of online course materials is used 
to keep students buying new temporary subscription services that used to 
be free. 
A professor can either choose a textbook from a catalog, one provided 
for them, or spend months to years writing one of their own. 
If a professor chooses to write their own textbook, 
they will have to seek publishers, other contributors, and have 
their work peer reviewed before it can finally be released to store 
shelves. 
If a professor chooses a book through a catalog or one provided for 
them, the pre-written textbooks often contain unnecessary chapters, 
or can even be missing key sections on topics viewed as critical. 
This problem can further result in additional class materials being 
purchased by students to supplement the missing information. \par}

%\vspace{1pc}
\subsection{Proposed Solutions}
\vspace{1pc}

{\noindent The Many Voices platform instead allows professors to create 
their own textbooks by using crowd-sourced content. 
Content creators are able to submit materials of all different formats, 
including: images, small pieces of text, pages, or even entire chapters. 
The submitted materials are able to be reviewed by editors and 
other contributors to prevent invalid materials from being used.
A knowledgeable professor can contribute a chapter or unit on 
their specialty, and others are able to modify obtained materials 
in the way they see fit. These materials may be offered for free 
or for a nominal fee, determined by the content creator. \par}

\medskip

{\noindent Once materials are acquired, anyone is able to create a more 
affordable textbook available in either a downloadable eBook or in 
printed form to create the exact textbook they want for their classes or 
for any purpose they desire. 
These materials are able to be hosted on a version control website such 
as GitHub, that allow for users to fork and branch their changes to 
satisfy their goals.
Additionally the platform will feature a way for editors to review 
content, and moderate changes in a manner similar to the approach 
Wikipedia has for high conflict pages, restricting who can edit for very 
popular pages. \par}

\medskip

{\noindent The platform will also feature a way to search for and find 
documents relevant to the users interests. 
Through a variety of search criteria, the requested topic or subject will
turn up results prioritized by relevance and credibility. By default, the 
platform will offer the most used and/or reviewed materials
based on entered search terms. Users will be able to narrow down their 
options by specifying date ranges, certain contributors, editor review 
score, and more choices to make finding the right document an easy 
process. 
The finished platform will allow users to see existing 
textbooks, customize them using their own or third party materials, and 
contribute their customizations back to other professors. \par}

%\vspace{1pc}
\subsection{Performance Metrics}
\vspace{1pc}

{\noindent The platform will be viewed as a success if the platform 
allows instructors to submit, edit, publish, and compile materials for 
the creation of textbooks and other instructional materials. 
Our goal is to ensure the platform meets our client's project 
expectations, the defined needs, and functions as according to our 
development plan. 
We will, in our best efforts, plan to meet at minimum every two weeks 
with our client to discuss current progress of project goals and 
proposed functionality for the application. 
Finally, before beginning work on major project milestones and certain 
aspects of the platform, we will discuss the overall status of the 
project. 
We will further inquire as to whether our client has any additional 
insight to assist in meeting project expectations as desired. 
After reaching a reasonable agreement, we will continue to adjust and 
shape the platform where necessary. 
We will further ensure that the development taking place on the platform 
is to the best of our abilities. \par}

\medskip

\newpage
\centerline{\sc \large Signature Page}
\vspace{5pc}


\centering

\begin{tabular}{lllll}
Dr. Carlos Jensen, Client    & \_\_\_\_\_\_\_\_\_\_\_\_\_\_\_\_\_\_\_\_\_\_\_\_\_\_\_\_\_\_\_\_\_\_ & Date & \_\_\_\_\_\_\_\_\_\_\_\_\_\_\_\_\_\_\_\_\_ &  \\
                         &                                                                                  &      &                                            &  \\ \\
Steven Powers, Developer & \_\_\_\_\_\_\_\_\_\_\_\_\_\_\_\_\_\_\_\_\_\_\_\_\_\_\_\_\_\_\_\_\_\_ & Date & \_\_\_\_\_\_\_\_\_\_\_\_\_\_\_\_\_\_\_\_\_ &  \\ 
                         &                                                                                  &      &                                            &  \\ \\
Josh Matteson, Developer & \_\_\_\_\_\_\_\_\_\_\_\_\_\_\_\_\_\_\_\_\_\_\_\_\_\_\_\_\_\_\_\_\_\_ & Date & \_\_\_\_\_\_\_\_\_\_\_\_\_\_\_\_\_\_\_\_\_ &  \\ 
                         &                                                                                  &      &                                            &  \\ \\
Evan Tschuy, Developer   & \_\_\_\_\_\_\_\_\_\_\_\_\_\_\_\_\_\_\_\_\_\_\_\_\_\_\_\_\_\_\_\_\_\_ & Date & \_\_\_\_\_\_\_\_\_\_\_\_\_\_\_\_\_\_\_\_\_ &  \\ 
                         &                                                                                  &      &                                            & 
\end{tabular}


\end{document}
